\documentclass[12pt]{article}
\usepackage{ctex}
\usepackage{array}
\usepackage{geometry}
\usepackage{graphicx}
\usepackage{fancyhdr}
\usepackage{ulem}
\usepackage{setspace}
\usepackage{lastpage}
\usepackage{fontspec}
%% \usepackage{fontawesome}
\usepackage{multirow}
\usepackage{multicol}
\usepackage{listings}
\usepackage{algorithm}
\usepackage{algorithmicx}
\usepackage{algpseudocode}
\usepackage{amsmath,amssymb}
\usepackage{epigraph}
\usepackage{hyperref}
\usepackage{tikz}
\hypersetup{
    colorlinks=true,
    linkcolor=blue,
    filecolor=blue,      
    urlcolor=blue,
    citecolor=cyan,
}
\floatname{algorithm}{算法}
\renewcommand{\algorithmicrequire}{\textbf{输入:}}
\renewcommand{\algorithmicensure}{\textbf{输出:}}
\geometry{a4paper,scale=0.8}
\setCJKmainfont[BoldFont=SimHei,ItalicFont=KaiTi]{SimSun}
\newfontfamily\tn{Times New Roman}
\title{\Huge C++基础~~期中考试}
\author{Jingheng Wang}
\date{2020 年 12 月 09 日}
\begin{document}
	\maketitle
	\thispagestyle{empty}
	\begin{center}
		\begin{tabular}{|p{0.20\textwidth}|p{0.20\textwidth}|p{0.20\textwidth}|p{0.20\textwidth}|}
		\hline
		题目名称 & 各位数之和 & 阶乘的末尾 & 进制转换 \\
		\hline
		题目类型 & 传统型 & 传统型 & 传统型 \\
		\hline
		提交文件名 & \verb|sumdigit.cpp| & \verb|factorial.cpp| & \verb|baseconv.cpp| \\
		\hline
		可执行文件名 & \verb|sumdigit| & \verb|factorial| & \verb|baseconv| \\
		\hline
		输入文件名 & \verb|标准输入输出| & \verb|标准输入输出| & \verb|标准输入输出| \\
		\hline
		输出文件名 & \verb|标准输入输出| & \verb|标准输入输出| & \verb|标准输入输出| \\
		\hline
		每个测试点时限 & $1.0\ \mathrm{sec}$ & $1.0\ \mathrm{sec}$ & $1.0\ \mathrm{sec}$ \\
		\hline
		内存限制 & $64\ \mathrm{MB}$ & $64\ \mathrm{MB}$ & $64\ \mathrm{MB}$ \\
		\hline
		测试点数量 & $10$ & $10$ & $10$ \\
		\hline
		单个测试点分值 & $10$ & $10$ & $10$ \\
		\hline
		结果比较方式 & 全文比较 & 全文比较 & 全文比较  \\
		\hline
		编译选项 & \verb|| & \verb|| & \verb|| \\
		\hline
		\end{tabular}\\

		\section*{注意事项}
		\begin{itemize}
		\item 文件名必须使用英文小写。
		\item C++中函数main()的返回值类型必须是int,程序正常结束时的返回值必须是0。
		\item 程序可使用的栈内存空间限制与题目的内存限制一致。
		\item 评测时采用的机器配置为:Intel(R) Core(TM) i7-7500U CPU @ 2.70GHz,内存16GB。上述时限以此为标准。
		\item 提交时应提交一个压缩包,压缩包内包含一个文件夹,文件夹内包含三个以.cpp为后缀的文件。
		\end{itemize}
	\end{center}
	
	\newpage
	\begin{center} \section*{\Large 各位数之和(sumdigit)} \end{center}
	\pagestyle{fancy}
	\fancyhf{}
	\fancyhead[L]{\footnotesize C++基础~~期中考试}
	\fancyhead[R]{\footnotesize 各位数之和(sumdigit)}
	\fancyfoot[C]{\footnotesize 第 \thepage 页\quad 共 \pageref{LastPage} 页}
	
	\subsection*{\normalsize 【问题描述】}
	输入一个非负整数,输出其各位数字之和。
	
	\subsection*{\normalsize 【输入格式】}
	一行,为一个非负整数$N$。
	
	\subsection*{\normalsize 【输出格式】}
	一行,为$N$各位上的数字之和。
	
	\subsection*{\normalsize 【样例 1 输入】}
	12345
	\subsection*{\normalsize 【样例 1 输出】}
	15
	\subsection*{\normalsize 【样例 1 解释】}
	$1+2+3+4+5 = 15$
	
	\subsection*{\normalsize 【样例 2 输入】}
	710385
	\subsection*{\normalsize 【样例 2 输出】}
	24
	\subsection*{\normalsize 【样例 2 解释】}
	$7+1+0+3+8+5 = 24$
	\subsection*{\normalsize 【数据规模与约定】}
	\begin{center}
		\begin{tabular}{|p{0.20\textwidth}<{\centering}|p{0.20\textwidth}<{\centering}|p{0.20\textwidth}<{\centering}|}
		\hline
		子任务编号 & $N$ & 其他限制\\
		\hline
		$1 \sim 3$ & $N \le 100$ & \multirow{4}{*}{无}\\
		\cline{1-2}
		$4 \sim 7$ & $N \le 10^5$ & \\
		\cline{1-2}
		$8 \sim 9$ & $N \le 10^9$ & \\
		\cline{1-2}
		$10$ & $N \le 10^{12}$ & \\
		\hline
		\end{tabular}\\
	\end{center}

	\newpage
	\begin{center} \section*{\Large 阶乘的末尾(factorial)} \end{center}
	\pagestyle{fancy}
	\fancyhf{}
	\fancyhead[L]{\footnotesize C++基础~~期中考试}
	\fancyhead[R]{\footnotesize 阶乘的末尾(factorial)}
	\fancyfoot[C]{\footnotesize 第 \thepage 页\quad 共 \pageref{LastPage} 页}
	
	\subsection*{\normalsize 【问题描述】}
	定义一个非负整数$N$的\textbf{阶乘}$N!$为:
$$
N~! = 1 \times 2 \times 3 \times \dots N ~(N \ge 1) \\
0! = 1
$$
	小清注意到,有一些阶乘的结果末尾有不少个0。比如,
$$
11! = 39916800
18! = 6402373705728000
$$
	小清想知道,对于给定的非负整数$N$,$N!$的末尾一共有多少个0呢?
	
	\subsection*{\normalsize 【输入格式】}
	一行,为一个非负整数$N$。
	
	\subsection*{\normalsize 【输出格式】}
	一行,为$N!$末尾0的个数。答案应为一个非负整数。
	
	\subsection*{\normalsize 【样例 1 输入】}
	18
	\subsection*{\normalsize 【样例 1 输出】}
	3
	\subsection*{\normalsize 【样例 1 解释】}
	$18! = 6402373705728000$,末尾有3个0。
	
	\subsection*{\normalsize 【样例 2 输入】}
	1300
	\subsection*{\normalsize 【样例 2 输出】}
	324
	\subsection*{\normalsize 【数据规模与约定】}
	\begin{center}
		\begin{tabular}{|p{0.20\textwidth}<{\centering}|p{0.20\textwidth}<{\centering}|p{0.20\textwidth}<{\centering}|}
		\hline
		子任务编号 & $N$ & 其他限制\\
		\hline
		$1 \sim 3$ & $N \le 12$ & \multirow{3}{*}{无}\\
		\cline{1-2}
		$4 \sim 7$ & $N \le 10^4$ & \\
		\cline{1-2}
		$8 \sim 10$ & $N \le 10^9$ & \\
		\hline
		\end{tabular}\\
	\end{center}

	\newpage
	\begin{center} \section*{\Large 进制转换(baseconv)} \end{center}
	\pagestyle{fancy}
	\fancyhf{}
	\fancyhead[L]{\footnotesize C++基础~~期中考试}
	\fancyhead[R]{\footnotesize 进制转换(baseconv)}
	\fancyfoot[C]{\footnotesize 第 \thepage 页\quad 共 \pageref{LastPage} 页}
	
	\subsection*{\normalsize 【问题描述】}
	给出一个十进制的非负整数$N$,一个进制$b$,请你输出在$b$进制下等同于十进制的$N$的数。
	
	\subsection*{\normalsize 【输入格式】}
	一行,为一个十进制非负整数$N$,和进制$b$,中间用一个空格隔开。
	
	\subsection*{\normalsize 【输出格式】}
	一行,为在$b$进制下的$N$。
	
	\subsection*{\normalsize 【样例 1 输入】}
	50 2
	\subsection*{\normalsize 【样例 1 输出】}
	110010
	\subsection*{\normalsize 【样例 1 解释】}
	$(110010)_2 = 2^5 + 2^4 + 2^1 = 32 + 16 + 2 = (50)_{10}$, 
	
	\subsection*{\normalsize 【样例 2 输入】}
	97 6
	\subsection*{\normalsize 【样例 2 输出】}
	241
	\subsection*{\normalsize 【样例 2 解释】}
	$(241)_6 = 2 \times 6^2 + 4 \times 6^1 + 1 \times 6^0 = 72 + 24 + 1 = (97)_{10}$

	\subsection*{\normalsize 【样例 3 输入】}
	736548641 8
	\subsection*{\normalsize 【样例 3 输出】}
	5731553155
	
	\subsection*{\normalsize 【数据规模与约定】}
	\begin{center}
		\begin{tabular}{|p{0.20\textwidth}<{\centering}|p{0.20\textwidth}<{\centering}|p{0.20\textwidth}<{\centering}|p{0.20\textwidth}<{\centering}|}
		\hline
		子任务编号 & $N$ & $b$ & 其他限制\\
		\hline
		$1$ & $N \le 100$ & $b = 10$ &\multirow{4}{*}{无}\\
		\cline{1-3}
		$2 \sim 4$ & $N \le 100$ & $ b = 2 $ & \\
		\cline{1-3}
		$5 \sim 7$ & $N \le 10^5$ &  \multirow{2}{*}{$2 \le b \le 10$} & \\
		\cline{1-2}
		$8 \sim 10$ & $N \le 10^9$ & &\\
		\hline
		\end{tabular}\\
	\end{center}
	
\end{document}
